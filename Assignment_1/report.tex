\documentclass{article}
\usepackage[utf8]{inputenc}

\usepackage{listings}
\lstset{
  basicstyle=\footnotesize,
  numbers=left, 
  stepnumber=1, 
  numbersep=5pt,
  showspaces=false,
  showstringspaces=false,
  showtabs=false,
  frame=single,
  tabsize=2,
  captionpos=b,
  breaklines=true,
}

\usepackage{pgfplots}

\usepackage{algorithm}  
\usepackage{algorithmicx}  
\usepackage{algpseudocode}  
\usepackage{amsmath}  
  
\floatname{algorithm}{algorithm}
\renewcommand{\algorithmicrequire}{\textbf{Input:}}  
\renewcommand{\algorithmicensure}{\textbf{Output:}}  

\title{Assignment 1}
\author{Name: Shiyin Lin \\UFID: 49619012}
\date{}

\begin{document}

\maketitle

\section{Cycle Finding}
\subsection{Pseudo-code}
\begin{algorithm}
    \caption{Disjoint-set data structure}
    \begin{algorithmic}[1] 
        \Require $Array$\ Edges  
        \Ensure bool
        \Function{DSU}{$Array$}
            \If {$x$ is not already in the forest}  
                \State $x.patent := x$
            \EndIf
            \For{each $edge$ in Edges}
                \State $pf := $\Call{Find}{edge.fr}
                \State $pt := $\Call{Find}{edge.to}
                \If {$pf = pt$}
                    \State $Find\ cycle$
                    \State \Return{$true$}
                \EndIf
                \State $y.parent := x$
            \EndFor
            \State $No\ cycle$
            \State \Return{$false$}
        \EndFunction
        \State  
        \Function{Find}{$x$} is 
            \While{$x.parent \neq x$}
                \State $(x, x.parent) := (x.parent, x.parent.parent)$  
            \EndWhile  
            \State \Return{$x$}  
        \EndFunction
    \end{algorithmic}  
\end{algorithm} 
\subsection{Proof of Correctness}
Assume\ the\ edge's\ node x and node y are in the set but cannot form a cycle.\\
node x and node y are in the set\\
$\Rightarrow$node x is connected to node y\\
node x and node y cannot form a cycle\\
$\Rightarrow$No edge contains node x and node y\\
$\Rightarrow$contradiction\\
$\Rightarrow$node x and node y in the set can form a cycle

\subsection{Algorithm's running time}
When initializing the parent array, the loop is $n$ times, and the complexity is $\Theta(n)$. If the lookup path contains $s$ nodes, it is obvious that the time complexity of the lookup is $\Theta(s)$. The time complexity of the find(a) and union(a) operation can be proven to be $\Theta(\alpha(n))$ if no node's potential energy increases due to optimization using path compression and at least $s-\alpha(n)$ nodes have their potential energy reduced by at least 1. So the total cost is $\Theta(n)+\Theta(\alpha(n))$


\subsection{Implement}
\subsubsection{Graph Generator}
\begin{lstlisting}
func generateUndirected(nnum int, enum int) graph.Undirected {
    r := rand.New(rand.NewSource(time.Now().Unix())
    return graph.GnmUnidirected(nnum, enum, r)
}

func generateEdges(g graph.Undirected) [][]int {
    edgeCollect := [][]int
    for fr, to := range g.AdjacencyList {
        for _, to := range to {
            if graph.NI(fr) < to {
                tmp := []int{}
                tmp = append(tmp, fr)
                tmp = append(tmp, int(to))
                edgeCollect = append(edgeCollect, tmp)
            }
        }
    }
    return edgeCollect
}
\end{lstlisting}

\subsubsection{Test Code}
\begin{lstlisting}
func Test() {
    nodes := 10000000
    edges := 20000000
    g := generateUndirected(nodes, edges)
    edgeCollect := generateEdges(g)
    start := time.Now()
    FindCycleDSU(edgeCollect)
    elapsed := time.Since(start)
    fmt.Println("total cost:", elapsed)
}
\end{lstlisting}

\subsubsection{Test algorithm for increasing graph size}
\begin{tabular}{cccccc}
\hline
Node & Edge & t1 & t2 & t3 & ave\\
\hline
10 & 20 & 25.025$\mu s$ & 25.826$\mu s$ & 26.6$\mu s$ &  25.817$\mu s$ \\
\hline
100 & 200 &  24.165$\mu s$ & 26.877$\mu s$ & 27.149$\mu s$ & 26.064$\mu s$ \\
\hline
1000 & 2000 &  34.612$\mu s$ & 32.773$\mu s$ & 33.821$\mu s$ & 33.735$\mu s$ \\
\hline
10000 & 20000 &  97.649$\mu s$ & 111.776$\mu s$ & 98.966 $\mu s$ & 102.797$\mu s$ \\
\hline
100000 & 200000 & 508.58$\mu s$ & 452.34$\mu s$ & 629.028$\mu s$ & 529.983$\mu s$ \\
\hline
1000000 & 2000000 & 7.098$ms$ & 8.127$ms$ & 8.371$ms$ & 7.865$ms$ \\
\hline
10000000 & 20000000 & 114.545$ms$ & 109.498$ms$ & 95.398$ms$ & 106.48$ms$ \\
\hline
\end{tabular}

\subsubsection{Plot running time}
\begin{tikzpicture}
\begin{axis}
\addplot+[sharp plot]
coordinates{ (0, 25.817) (1, 26.064) (2, 33.735) (3, 102.797) (4, 529.983) (5, 7865) (6, 106480) };
\end{axis}
\end{tikzpicture}

\newpage
\section{Minimum Spanning Tree for sparse graphs}
\subsection{Pseudo-code}
\begin{algorithm}
\caption{Kruskal}
    \begin{algorithmic}[1]
        \Require $Int$\ numOfNodes, $Int$\ numOfEdges, $Array$\ EdgesWithWright
        \Ensure $Null$
        \Function{Kruskal}{$Int, Int, Array$}
            \If {$node$ is not already in the forest}  
                \State $node.patent := node$
            \EndIf
            \State \Call{QuicksSore}{Edges}
            \While{$i < $ numOfEdges}
                \State $pf := $\Call{Find}{edge[i].fr}
                \State $pt := $\Call{Find}{edge[i].to}
                \If {$pf \neq pt$}
                    \State  $pt.parent = pf$
                    \State $resault += edge.weight$
                \EndIf
            \EndWhile
            \If {not $result$}
                \State $Can't\ find\ MST$
            \Else
                \State $MST\ is:\ result$
            \EndIf
        \EndFunction
    \end{algorithmic} 
\end{algorithm} 

\subsection{Proof of Correct}
Suppose that the graph $G$ has $n$ vertices, then the spanning tree must have $n-1$ edges. Since the number of spanning trees of a graph is finite, there is at least one tree with minimum cost, which is assumed to be $U$. First make the following assumptions.\\

1) The output tree is $T$.\\

2) The number of different edges in $U$ and $T$ is $k$, and the other $n-1-k$ edges are the same, and these $n-1-k$ edges form the edge set $E$.\\

3) The edges in $T$ but not in $U$ are $a_1, a_2, ...$ in order of cost from smallest to largest. , $a_k$.\\

4) The edges in $U$ but not in $T$ are $x_1, x_2, ...$ in order of cost from smallest to largest. , $x_k$.\\

By converting $U$ to $T$ (moving the edges of $T$ into $U$ in order), to prove that $U$ and $T$ have the same cost.\\

First, we move $a_1$ into $U$. Since $U$ itself is a tree, adding any edge at this point constitutes a cycle, so the addition of $a_1$ must produce a cycle, and this cycle must include the edges in $x_1, x_2, ... , x_k. , x_k.$ (Otherwise $a_1$ and the edges in $E$ form a cycle, and $E$ is also in $T$, which contradicts the absence of a cycle in $T$.)\\

In this cycle delete edges belonging to $x_1, x_2, ... , x_k$ and the most costly edge $x_i$ forms a new spanning tree $V$.\\

Assuming that $a_1$ cost is less than $x_i$, the cost of $V$ is less than $U$. This contradicts that $U$ is a minimum cost tree, so $a_1$ cannot be less than $x_i$.\\
Assuming that $a_1$ is greater than $x_i$, according to Kruskal's algorithm, the edge with small cost is considered first, then when Kruskal's algorithm is executed, $x_i$ should be considered before $a_1$, which in turn is considered before $a_2, ... , a_k$ before considering $x_i$, so before considering $x_i$, the edges in $T$ can only be edges in $E$. And since $x_i$ did not join the tree $T$, it means that $x_i$ must constitute a cycle with some edges in $E$. But $x_i$ and $E$ are in $U$ at the same time, which contradicts with $U$ being a spanning tree, so $a_1$ cannot be larger than $x_i$ either.\\

Therefore, the newly obtained tree $V$ has the same cost as $T$.\\

By analogy, adding the edges of $a_1, a_2, ... ,$ the edges of $a_k$ are gradually added to $U$, and the final obtained tree $T$ has the same cost as $U$.

\subsection{Algorithm's running time}
Since the algorithm is optimized using the DSU of path compression, the complexity is $\Theta(n)+\Theta(\alpha(n))$, and since it contains a quick sort operation, the consultation complexity is $\Theta(n)+\Theta(\alpha(n))+\Theta(mlogm)$

\subsection{Implement}
\subsubsection{Graph Generator}
\begin{lstlisting}
func generateConnectGraph(nnum int, enum int) graph.Undirected {
    n, _ := rand2.Int(rand2.Reader, big.NewInt(65536) 
	r := rand.New(rand.NewSource(n.Int64()))
	g := graph.GnmUndirected(nnum, enum, r)
	if !g.IsConnected() {
		generateConnectGraph(nnum, enum)
	}
	return g
}

func generateWEdges(g graph.Undirected, enum int) [][]int {
	weight := make([]int, enum)

	for i := 0; i < enum; i++  {
		n, _ := rand2.Int(rand2.Reader, big.NewInt(int64(20)))
		if n.Int64() == 0 {
			weight[i] = int(n.Int64()) + 1
		} else {
			weight[i] = int(n.Int64())
		}
	}
	edgeCollect := [][]int{}
	for fr, to := range g.AdjacencyList {
		for _, to := range to {
			if graph.NI(fr) < to {
				tmp := []int{}
				tmp = append(tmp, fr)
				tmp = append(tmp, int(to))
				edgeCollect = append(edgeCollect, tmp)
			}
		}
	}
	for i:=0; i<enum; i++ {
		edgeCollect[i] = append(edgeCollect[i], weight[i])
	}
	return edgeCollect
}
\end{lstlisting}
\subsubsection{Test Code}
\begin{lstlisting}
func Test2() {
	nodes := 10
	edges := 18
	if edges < nodes-1 {
		fmt.Println("too less edges")
		os.Exit(-1)
	}
	g := generateConnectGraph(nodes, edges)
	edgeCollect := generateWEdges(g, edges)
	fmt.Println(edgeCollect)
	sort.Slice(edgeCollect[:edges], func(i,j int) bool {
		return edgeCollect[i][2] < edgeCollect[j][2]
	})
	fmt.Println(edgeCollect)
	start := time.Now()
	kruskal(nodes, edges, edgeCollect)
	elapsed := time.Since(start)
	fmt.Println("total cost:", elapsed)
}
\end{lstlisting}

\subsubsection{Test algorithm for increasing graph size}
\begin{tabular}{cccccc}
\hline
Node & Edge & t1 & t2 & t3 & ave\\
\hline
10 & 18 & 4.048$\mu s$ & 3.732$\mu s$ & 5.301$\mu s$ &  4.36$\mu s$ \\
\hline
20 & 28 &  5.059$\mu s$ & 5.263$\mu s$ & 5.503$\mu s$ & 5.275$\mu s$ \\
\hline
30 & 38 &  5.912$\mu s$ & 6.125$\mu s$ & 5.426$\mu s$ & 5.821$\mu s$ \\
\hline
40 & 48 &  5.534$\mu s$ & 6.11$\mu s$ & 6.466$\mu s$ & 6.037$\mu s$ \\
\hline
50 & 58 & 9.591$\mu s$ & 9.011$\mu s$ & 9.313$\mu s$ & 9.305$\mu s$ \\
\hline
60 & 68 & 9.369$\mu s$ & 9.515$\mu s$ & 9.561$\mu s$ & 9.482$\mu s$ \\
\hline
70 & 78 & 10.076$\mu s$ & 11.741$\mu s$ & 12.044$\mu s$ & 11.287$\mu s$ \\
\hline
\end{tabular}

\subsubsection{Plot running time}
\begin{tikzpicture}
\begin{axis}
\addplot+[sharp plot]
coordinates{ (0, 4.36) (1, 5.275) (2, 5.821) (3, 6.037) (4, 9.305) (5, 9.482) (6, 11.287) };
\end{axis}
\end{tikzpicture}

\end{document}
